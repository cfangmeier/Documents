\documentclass{article}
\usepackage{fontspec}
\usepackage{amsmath}
\usepackage{marvosym}
\usepackage{graphicx}
\usepackage{fullpage}
\setmainfont{DejaVu Serif}

\begin{document}
\thispagestyle{empty}
\textbf{PHYS 211 - Team Problem 10}
\hrule
\vspace{.3in}
It has been proposed that a good way to dispose of unwanted waste products(eg. nuclear waste)
would be to launch them into space. Ignoring the cost of getting the waste into orbit around the
sun, we want to know if it would be more efficient to send the waste out of the solar system or into the sun.

\begin{enumerate}
  \item Assuming we have an object of mass $m$ in a circular orbit around the sun with a radius $R$, calculate the work needed to
    \begin{enumerate}
      \item send the object out of the solar system. Remember, the object is trapped if its total energy is negative.
      \item send the object into the sun.
    \end{enumerate}
  \item In space flight, the real quantity of importance is not actually work, but $\Delta v$, or the change in velocity of the object. This is because a rocket with a certain amount of fuel can give the ship a fixed change in velocity, regardless of the initial velocity. Calculate the $\Delta v$ for the above two processes and determine which one is smaller. Does the result depend on $m$ or $R$?
\end{enumerate}

\emph{Useful Formula}
\begin{equation*}
  V_g(r) = -G\frac{mM}{r}
\end{equation*}
\begin{equation*}
  \vec{F}_g(r) = -G\frac{mM}{r^2}\hat{r}
\end{equation*}
\begin{equation*}
  a_{cent} = \frac{v^2}{r}
\end{equation*}

\end{document}
