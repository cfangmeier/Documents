\documentclass{beamer}
\usepackage{graphicx}
\usepackage{color}
\usepackage{fancybox}
\usepackage{appendixnumberbeamer}

\usetheme[left,hideothersubsections,height=1.2cm,width=2.0cm]{UNLTheme}

\title[\textbf{Address Pads}]{-- SiLab Lecture Series -- \\ \textbf{Address Pads}}
\author{Caleb~Fangmeier}
\institute[UNL]{University of Nebraska - Lincoln}
\date{June 2, 2016}
\subject{Address Pads}


\AtBeginSection[]
{
  \begin{frame}<beamer>{Outline}
    \tableofcontents[currentsection,currentsubsection]
  \end{frame}
}

\begin{document}

\begin{frame}
  \titlepage
\end{frame}

\section{Addressing}
\begin{frame}{Addressing}
    \begin{center}
    \alert{Problem}: \\ Many devices must communicate on shared lines.
    \pause
    \vspace{.15in}\hrule
    \end{center}
    \emph{Why?}
    \begin{itemize}
        \pause \item Limited physical space
        \pause \item Cost
        \pause \item Simplicity of design
    \end{itemize}
\end{frame}

\begin{frame}{Addressing}
    \centering
    \alert{Solution}: \\ Addressing!
    \pause 
    \vspace{.15in}\hrule
    \begin{itemize}
        \item Devices take turns talking on shared line
        \pause \item Typically controlled by a single ``\textbf{Master}''
        \pause \item All other devices are ``\textbf{Slaves}''
        \pause \item All \textbf{Slave} devices have an \textbf{address}
    \end{itemize}
\end{frame}

\section{A Simple Example}
\begin{frame}{The ``D'' Flip-Flop}
    \begin{columns}
        \begin{column}{0.33\textwidth}
            {\footnotesize
            \begin{itemize}
                \item Stores a single bit
                \item Has a \textbf{Data} input and a \textbf{Clock} input
                \item Data updates when clock transitions from low to high
            \end{itemize}
            }
        \end{column}
        \begin{column}{0.66\textwidth}
            \begin{figure}
                \shadowbox{
                \includegraphics[width=0.95\textwidth]{"figures/Master-Slave 1-1a"}
                }
            \end{figure}
        \end{column}
    \end{columns}
    \pause
    \vspace{.1in}
    \hrule
    But what if we want to control \emph{multiple} devices with a shared data and clock line?
\end{frame}

\begin{frame}{Shared Lines}
    \begin{columns}
        \begin{column}{0.33\textwidth}
            {\footnotesize
            \begin{itemize}
                \item Add an \textbf{Enable} line for each device
                \item Requires $n+2$ lines
                \item Data updates only when the \textbf{Enable} line is high
            \end{itemize}
            }
        \end{column}
        \begin{column}{0.66\textwidth}
            \begin{figure}
                \shadowbox{
                \includegraphics[width=0.95\textwidth]{"figures/Master-Slave 1-2a"}
                }
            \end{figure}
        \end{column}
    \end{columns}
    \pause
    \vspace{.1in}
    \hrule
    Cool, but is this \emph{optimal?} Short answer: It depends.
\end{frame}

\begin{frame}{Address Lines}
    \begin{columns}
        \begin{column}{0.33\textwidth}
            {\footnotesize
            \begin{itemize}
                \item Replace individual \textbf{Enable} lines with \textbf{Select}
                \item Requires $\lceil\log_2(\mathrm{n})\rceil+2$ lines
                \item In this case \textbf{A} has address ``$0$'', while \textbf{B} has address ``$1$''
                \item This scheme can be extened to any number of bits
            \end{itemize}
            }
        \end{column}
        \begin{column}{0.66\textwidth}
            \begin{figure}
                \shadowbox{
                \includegraphics[width=0.95\textwidth]{"figures/Master-Slave 1-2b"}
                }
            \end{figure}
        \end{column}
    \end{columns}
\end{frame}

\begin{frame}{2-wire Protocol}
    \begin{itemize}
        \item Address data is now pushed along data line
        \item Requires only $2$ lines
        \item In this case \textbf{A} has a 2-bit address (A1,A0)
        \item This diagram is incomplete since it doesn't include end-of-write circuitry, but it demonstrates the idea.
    \end{itemize}
    \begin{figure}
        \shadowbox{
        \includegraphics[width=0.75\textwidth]{"figures/Master-Slave 1-2c"}
        }
    \end{figure}
\end{frame}

\section{Pixel Address Pads}

\begin{frame}{CMS Pixel DAQ}
    \begin{figure}
        \centering
        \includegraphics[width=\textwidth]{"figures/Pixel Readout"}
    \end{figure}
\end{frame}

\begin{frame}{TBM08}
    \begin{columns}
        \begin{column}{0.45\textwidth}
            {\scriptsize
            \begin{tabular}{c|l}
                 \texttt{HA*} & Hub Address Pads \\
                 & Internally pulled down \\
                 \texttt{ClkIn} & Serial Clock Input \\
                 \texttt{SDaIn }& Serial Data  Input\\
                 \texttt{RCL} & Return Serial Clock \\
                 \texttt{RDa} & Return Serial Data \\
            \end{tabular}
            }
            \vspace{.1in}
            \hrule
            \vspace{.1in}
            Note: The \textbf{data} readout does \emph{not} happen on these lines. That is on \texttt{OutA}.
        \end{column}
        \begin{column}{0.55\textwidth}
            \begin{figure}
                \includegraphics[angle=90, height=0.80\textheight]{"figures/TBM08"}
            \end{figure}
        \end{column}
    \end{columns}
\end{frame}

\begin{frame}
    \begin{columns}
        \begin{column}{0.25\textwidth}
            \begin{figure}
                \includegraphics[angle=90, height=0.95\textheight]{"figures/Module"}
            \end{figure}
        \end{column}
        \begin{column}{0.75\textwidth}
            \begin{figure}
                \includegraphics[width=\textwidth]{"figures/review_address_HV_pads"}
            \end{figure}
        \end{column}
    \end{columns}
\end{frame}

\begin{frame}{Serial Command Prototcol}
    \begin{figure}
        \includegraphics[width=1.02\textwidth]{"figures/SerialCommandWaveform"}
    \end{figure}
\end{frame}

\begin{frame}{References}
    \nocite{*}
    \bibliographystyle{abbrv}
    {\tiny
    \bibliography{references} 
    }
\end{frame}



\end{document}
