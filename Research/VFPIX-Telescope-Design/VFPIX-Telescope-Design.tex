% arara: xelatex: {synctex: "1"}
% arara: bibtex
% arara: xelatex: {synctex: "1"}
% arara: xelatex: {synctex: "1"}

\documentclass{article}
\usepackage[affil-it]{authblk}
\usepackage{graphicx}
\usepackage[space]{grffile}
\usepackage{latexsym}
\usepackage{amsfonts,amsmath,amssymb}
\usepackage{url}
\usepackage[utf8]{inputenc}
\usepackage{hyperref}
\hypersetup{colorlinks=false,pdfborder={0 0 0}}
\usepackage{textcomp}
\usepackage{longtable}
\usepackage{multirow,booktabs}
\usepackage{fullpage}


\newcommand{\itemt}[1]{\item \textbf{#1}}

\begin{document}

\title{VFPIX Silicon Telescope \\ Design Document}
\author{Caleb Fangmeier \\
        Frank Meier}
\affil{Univ. of Nebraska-Lincoln}
\date{\today}


\maketitle

\begin{abstract}
  A silicon strip detector based telescope is being designed for the purpose of testing a under-development silicon pixel detector for the VFPIX upgrade of CMS.

  Here is documented the goals, constraints, and design decisions of the telescope.
\end{abstract}

\newpage

\tableofcontents

\newpage

\section{Introduction}

\section{Overview of Previous Work}
Here is some junk\cite{Turner2012}
\section{System Components}
The telescope consists of the following main components.
\begin{enumerate}
  \itemt{Silicon Strip Sensor}
  \itemt{Analog Pipeline Chip x128 (APC128)}
  \itemt{Sensor Mount Card (SMT)}
  \itemt{Back-Plane Board (BPB)}
  \itemt{Telescope Readout Board (TRB)}
\end{enumerate}

\section{Performance Targets}
\subsection{Precision/Accuracy Targets}
\subsection{Speed Targets}

\section{Hardware Proposal}

\section{Timeline}


\bibliographystyle{plain}
\bibliography{references}


\end{document}
