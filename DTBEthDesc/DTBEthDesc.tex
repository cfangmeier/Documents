\documentclass{article}
\usepackage{fullpage}
\usepackage{fontspec}
\setmainfont{Nimbus Roman}
\title{Pixel DTB Ethernet Communication Protocol (DECC)}
\author{Caleb Fangmeier \\ University of Nebraska, Lincoln}

\begin{document}
\maketitle
\tableofcontents

\section{Introduction}
This paper gives a description of the DTB Ethernet Communication and Control
(DECC) Protocol. The CMS experiment at CERN's LHC uses a tracker at its
innermost later. This tracker is made up of individual silicon detectors. For
the testing and comissioning of these detectors, a data aquisition(DAQ) device
was created. The current incarnation of the DAQ card is known as the Digital
TestBoard or DTB. 

The DTB can communicate directly with a silicon detector for
purposes such as reading out data, or configuration. The DTB is not designed to
interact with humans directly. Rather, it is connected to a PC via either
High Speed USB(USB 2.0) or Gigabit Ethernet. The PC then has software that
allows a user to request for a test to be run, and then communicates that
request to the DTB. 

The protocol that was chosen to facilitate communication between a PC and a DTB
was a custom flavor of Remote Procedure Call(RPC). The main idea of RPC is to
abstract the actual serial communication between a master(PC) and slave(DTB) as
a series of function calls. In other words, a user will call a function with
possible arguments and a return type on a the PC, then a matching function will
execute on the DTB. 

For the purposes of developing DECC, what matters about RPC is that it is just a
bidirectional serial stream of bytes. The details of how these bytes are
organized has no implications on the design of DECC. 
\section{System Requirements}
Based on the above considerations, the following requirements for DECC were laid
out.
\begin{enumerate}
  \item Speed is paramount. Therefore, DECC must have a minimal overhead. This
  is interpreted to mean two things.
  \begin{enumerate}
    \item The percent of data packets that is made up of bytes dedicated to DECC
    must be small, at least for large packets(\textgreater100 bytes).
    \item The computational overhead, especially on the DTB, must be minimal.
  \end{enumerate}
  \item DECC must allow for communication over a local subnet(eg. through an
  Ethernet switch or router). However, communication over a WAN or the Internet
  is not required. Therefore, an implementation of TCP/IP or UDP/IP is not
  necessary. However,
  \begin{enumerate}
    \item Multiple DTBs must be able to exist on a subnet without interfering
    with each other's operations.
    \item Multiple host PCs must also be able to exist on a single subnet
    without interfering with each other.
    \item Finally, a PC may control multiple DTBs on a subnet, but a DTB can
    only be controlled by a single PC at a time.
  \end{enumerate}
\end{enumerate}

\section{The DECC Protcol}
An Ethernet packet utilizing the DECC protocol will look like the following
\begin{table}[h]
\centering
\begin{tabular}{|l|c|c|c|c|c|c|c|}\hline
Field & DST-MAC & SRC-MAC & Ethertype & ProcNum & T-Byte & DatSize
& data \\ \hline 
Size in Bytes & 6 & 6 & 2 & 2 & 1 & 2 & 1-1500 \\ \hline
\end{tabular}
\end{table}

\begin{itemize}
  \item \textbf{DST-MAC} The MAC address of the destination device. Set to
  \verb!0xFFFFFFFFFFFF! for broadcast packets.
  \item \textbf{SRC-MAC} The MAC address of the source device. 
  \item \textbf{Ethertype} This identifies the protocol that is contained by the
  packet. For DECC, the Ethertype is \verb!0x0809!.
  \item \textbf{ProcNum} This is the process number of the PC application. 
  \item \textbf{T-Byte} This is used by DECC to identify what this packet is
  trying to do. See below.
  \item \textbf{DatSize} The size in bytes of the RPC packet.
  \item \textbf{data} The data that DECC is responsible for sending. Normally
  limited to 1500 bytes by the Ethernet standard.
\end{itemize}

The Type-Byte, or T-Byte, is responsible for identifying the type of a DECC
packet. The different possible values for T-Bytes and their meanings are
summarized in the following table.

\begin{table}[h]
\centering
\begin{tabular}{|l|l|l|} \hline
T-Byte  & Sent from PC     & Sent from DTB \\ \hline 
0x0 & RPC Packet           & RPC Packet \\ \hline
0x1 & Hello: Query DTBs on subnet for status        & status: unclaimed, or
operation: sucessful \\ \hline
0x2 & Claim: Attempt to claim targeted DTB          & status: claimed, or
operation: failure   \\ \hline
0x3 & Unclaim: Attempt to free the targeted DTB     & UNUSED\\ \hline
0x4 & Force Claim: Claim DTB, even if not unclaimed & UNUSED \\ \hline
\end{tabular}
\end{table}


The way the DECC protocol works is the following. The initial state of the
system is there are a set of DTBs and a set of PCs on a local subnet. All of the
DTBs are unclaimed. PC A sends out a ``Hello'' DECC packet which is broadcast
throughout the subnet. Every DTB will then send a response with it's status:
either claimed or unclaimed. Since all DTBs are unclaimed, the T-Byte of all
response packets will be 0x1, meaning ``unclaimed''. The user at PC A can then
select from a list of available DTBs one to connect to, or ``claim''. PC A will 
then send out another DECC packet with T-Byte 0x2 to the selected DTB. If the
DTB hasn't been claimed in the meantime by another PC, the DTB will accept the
claim and send back a response packet with T-Byte 0x1, meaning the claim was
sucessful. If the claim was unsucessful, the T-Byte will be 0x2. If the claim
was sucessful, PC A and the claimed DTB can now communicate RPC commands until
finished. When finished, PC A will send an unclaim packet to the DTB. It is then
available to be claimed by another PC. 

As another example, say there are two DTBs and a single PC on a subnet. The
operator wants to run tests with both DTBs simultaneously with a single PC. This
is accomplished using the ProcNum field. When a DTB is claimed, it remembers
both the MAC address of the claiming PC as well as the ProcNum of the claiming
program on that PC. The process number is simply the default, ProcNum can be any
identifying number. (For example, in the case where multiple DTBs need to be
controlled by a single instance of a program.) Now, when a DTB sends data to
it's controlling \textit{process}, it addresses the packet both to the MAC
address of the controlling machine and the process number on that machine.
Together this data uniquely identify the recipient of the packet.


\end{document}
